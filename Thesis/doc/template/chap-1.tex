\chapter{The First Chapter}
\label{cha:1}
A chapter is a logical unit. It normally starts with an introduction, which
you are reading now. The last topic of the chapter holds the conclusion.

\section{The First Topic of the Chapter}
First comes the introduction to this topic.

\lipsum[55]

\subsection{An item}
Please don't abuse enumerations: short enumerations shouldn't use
``\verb|itemize|'' or ``\texttt{enumerate}'' environments.
So \emph{never write}: 
\begin{quote}
  The Eiffel tower has three floors:
  \begin{itemize}
  \item the first one;
  \item the second one;
  \item the third one.
  \end{itemize}
\end{quote}
But write:
\begin{quote}
  The Eiffel tower has three floors: the first one, the second one, and the
  third one.
\end{quote}

\section{A Second Topic}
\lipsum[64]

\subsection{Another item}
\lipsum[56-57]

\section{Conclusion}
The final section of the chapter gives an overview of the important results
of this chapter. This implies that the introductory chapter and the
concluding chapter don't need a conclusion.

\lipsum[66]

%%% Local Variables: 
%%% mode: latex
%%% TeX-master: "thesis"
%%% End: 
